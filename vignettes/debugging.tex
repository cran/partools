\documentclass[11pt]{article}\usepackage[]{graphicx}\usepackage[]{color}
%% maxwidth is the original width if it is less than linewidth
%% otherwise use linewidth (to make sure the graphics do not exceed the margin)
\makeatletter
\def\maxwidth{ %
  \ifdim\Gin@nat@width>\linewidth
    \linewidth
  \else
    \Gin@nat@width
  \fi
}
\makeatother

\definecolor{fgcolor}{rgb}{0.345, 0.345, 0.345}
\newcommand{\hlnum}[1]{\textcolor[rgb]{0.686,0.059,0.569}{#1}}%
\newcommand{\hlstr}[1]{\textcolor[rgb]{0.192,0.494,0.8}{#1}}%
\newcommand{\hlcom}[1]{\textcolor[rgb]{0.678,0.584,0.686}{\textit{#1}}}%
\newcommand{\hlopt}[1]{\textcolor[rgb]{0,0,0}{#1}}%
\newcommand{\hlstd}[1]{\textcolor[rgb]{0.345,0.345,0.345}{#1}}%
\newcommand{\hlkwa}[1]{\textcolor[rgb]{0.161,0.373,0.58}{\textbf{#1}}}%
\newcommand{\hlkwb}[1]{\textcolor[rgb]{0.69,0.353,0.396}{#1}}%
\newcommand{\hlkwc}[1]{\textcolor[rgb]{0.333,0.667,0.333}{#1}}%
\newcommand{\hlkwd}[1]{\textcolor[rgb]{0.737,0.353,0.396}{\textbf{#1}}}%

\usepackage{framed}
\makeatletter
\newenvironment{kframe}{%
 \def\at@end@of@kframe{}%
 \ifinner\ifhmode%
  \def\at@end@of@kframe{\end{minipage}}%
  \begin{minipage}{\columnwidth}%
 \fi\fi%
 \def\FrameCommand##1{\hskip\@totalleftmargin \hskip-\fboxsep
 \colorbox{shadecolor}{##1}\hskip-\fboxsep
     % There is no \\@totalrightmargin, so:
     \hskip-\linewidth \hskip-\@totalleftmargin \hskip\columnwidth}%
 \MakeFramed {\advance\hsize-\width
   \@totalleftmargin\z@ \linewidth\hsize
   \@setminipage}}%
 {\par\unskip\endMakeFramed%
 \at@end@of@kframe}
\makeatother

\definecolor{shadecolor}{rgb}{.97, .97, .97}
\definecolor{messagecolor}{rgb}{0, 0, 0}
\definecolor{warningcolor}{rgb}{1, 0, 1}
\definecolor{errorcolor}{rgb}{1, 0, 0}
\newenvironment{knitrout}{}{} % an empty environment to be redefined in TeX

\usepackage{alltt}

\setlength{\oddsidemargin}{0.0in}
\setlength{\evensidemargin}{0.0in}
\setlength{\topmargin}{-0.25in}
\setlength{\headheight}{0in}
\setlength{\headsep}{0in}
\setlength{\textwidth}{6.5in}
\setlength{\textheight}{9.25in}
\setlength{\parindent}{0in}
\setlength{\parskip}{2mm} 

% \usepackage{times}
\usepackage{graphicx}

% library(knitr)
%\VignetteIndexEntry{Partools}

\title{{\bf Debugging the partools and parallel Packages}}

\author{Norm Matloff \\
University of California, Davis}
\IfFileExists{upquote.sty}{\usepackage{upquote}}{}
\begin{document}

\maketitle

I place huge importance on debugging, and indeed, once wrote a book on
the topic ({\it The Art of Debugging with GDB, DDD, and Eclipse}, N.
Matloff and P. Salzman, NSP, 2008). Accordingly, {\bf partools} includes
some debugging tools, which by the way apply not only to {\bf partools}
but to the {\bf parallel} package in general.  This vignette will
introduce the usage of those tools.

\section{Running Example}

We'll use this simple file, {\bf fg.R} as our example:

\begin{verbatim}
f <- function(x) {
   x <- x + 1
   y <- g(x)
   x^2 + y^2
}

g <- function(t) {
   if (t > 0) return(5)
   6
}
\end{verbatim}

\section{The Fundamental Principle of Debugging}

Pete Salzman and I make this statement, early in our book on debugging:

\begin{quote}
Fixing a buggy program is a process of confirming, one by one, that the
many things you {\it believe} to be true about the code actually {\it
are} true.  When you find that one of your assumptions is {\it not}
true, you have found a clue to the location (if not the exact nature) of
a bug.
\end{quote} 
In our code above, after executing

\begin{verbatim}
y <- g(x)
\end{verbatim}

we may believe that {\bf y} should be 5. But we should check that it
really is 5. Eventually, we will find a case where our belief is wrong,
and thus have a clue about the bug.

\section{Quick and Dirty Method}

The classic way to do the confirmation discussed in the last section is
to add code, say that prints out the values of some key variables. This
is not generally recommended, because the process of inserting such
check code and later removing it is too distracting, making us lose our
train of thought.  It's much preferable to use a debugging aid, even a
primitive one like R's {\bf debug()}, than to insert print statements.
We'll use {\bf debug()} in the next section.

But even adding print statements, which we will do in this section, is
not so easy as it sounds, because in {\bf parallel} operations, we don't
have terminal windows.  A statement such as

\begin{verbatim}
cls <- makeCluster(2)
\end{verbatim}

sets up 2 instantiations of R, but they are not attached to terminal
windows, so we can't invoke {\bf debug()}!

What we can do instead, of course, is insert code that writes
confirmation messages to a file rather than to the sccreen.  Even that
is not so easy, because if all our cluster nodes write to the same file,
we have chaos.

Instead, {\bf partools} includes a function {\bf dbsmsg()} that does
print to a file, but with the file name having a suffix corresponding to
the cluster node number.  With a 2-node cluster, the file names would be
{\bf dbs.1} and {\bf dbs.2}.  We run the code, then check the files.

In the little example above, say, we add the call to {\bf dbsmsg()} and
get the code to the cluster nodes:

\begin{verbatim}
> source('fg.R')
> f
function(x) {
   x <- x + 1
   y <- g(x)
   dbsmsg(y)
   x^2 + y^2
}
> clusterExport(cls,c('f','g'))
\end{verbatim}

Then run:

\begin{verbatim}
> clusterEvalQ(cls,f(8))
\end{verbatim}

We then check the files {\bf dbs.1} and {\bf dbs.2} to see the value of
{\bf y} (both 5 in this simple example).

A similar, but somewhat more sophisticated method involves the function
{\bf dbsdump()}.  We add the call to, say, {\bf g()}, then run:

\begin{verbatim}







> clusterExport(cls,'g')
> clusterEvalQ(cls,f(8))
\end{verbatim}

This creates one output file for each node.  Let's use the one from node
2:

\begin{verbatim}
> load('last.dump.2.rda')
> debugger(last.dump.2)
Message:  Available environments had calls:
1: parallel:::.slaveRSOCK()
2: slaveLoop(makeSOCKmaster(master, port, timeout, useXDR))
3: tryCatch({
    msg <- recvData(master)
    if (msg$type == "DONE") {
        c
4: tryCatchList(expr, classes, parentenv, handlers)
5: tryCatchOne(expr, names, parentenv, handlers[[1]])
6: doTryCatch(return(expr), name, parentenv, handler)
7: tryCatch(do.call(msg$data$fun, msg$data$args, quote = TRUE), error = handle
8: tryCatchList(expr, classes, parentenv, handlers)
9: tryCatchOne(expr, names, parentenv, handlers[[1]])
10: doTryCatch(return(expr), name, parentenv, handler)
11: do.call(msg$data$fun, msg$data$args, quote = TRUE)
12: (function (expr, envir = parent.frame(), enclos = if (is.list(envir) || is.
13: eval(expr, envir, enclos)
14: f(8)
15: <tmp>#4: dbsdump()
16: eval(parse(text = cmd))
17: eval(expr, envir, enclos)

Enter an environment number, or 0 to exit  Selection: 14
Browsing in the environment with call:
   f(8)
Called from: debugger.look(ind)
Browse[1]> x
[1] 9
Browse[1]> y
[1] 5
Browse[1]> debug(g)
Browse[1]> g(x)
debugging in: g(x)
debug at <tmp>#1: {
    if (t > 0) 
        return(5)
    6
}
Browse[4]> t
[1] 9
\end{verbatim}

We checked the values of {\bf x} and {\bf y} in {\bf f()}, and then even
made a call to {\bf g()}, checking things out there.

Again, this is an exceedingly simple example, but applying these ideas
in a more realistic setting would involve the same actions.

\section{Advanced Method}

This method is suitable for more detailed debugging that requires
single-stepping through code and so on. It is somewhat more involved,
but actually very easy to use after doing it once.  It requires a
Unix-family system, such as a Mac or a Linux box, or Cygwin on Windows,
because it make use of the {\bf screen} utility.  It allows you to debug
your code with the ordinary R {\bf debug()} facility, with one
instantiation of the latter for each of your cluster nodes.

So, here is our initial screen setting:

% \begin{figure}
\includegraphics[width=6.5in]{dbs1.png}
% \end{figure}

In preparation for running {\bf debug()} on two cluster nodes, we have
the two windows on the left.  To run the ``manager'' node, from which
{\bf parallel} functions will be called such as

\begin{verbatim}
> clusterEvalQ(cls,f(8))
\end{verbatim}

we have another window on the top right. {\bf Under Linux, the
creation of these windows can be done automatically.}

We launch the debugging from the partially hidden window:

% \begin{figure}
\includegraphics[width=4.5in]{dbs2.png}
% \end{figure}

The call to {\bf dbs()} sets up a {\bf parallel} cluster {\bf cls} with
2 nodes, and instructs those nodes to do {\bf source('fg.R')} and {\bf
debug(f)}.  It also instructs us, the user, to run {\bf screen} in the
manager window and in each cluster window, and it starts up R for us in
the manager window:

\includegraphics[width=6.5in]{dbs3.png}

Finally, {\bf dbs()} runs R in the two cluster windows.  Those
instantiations of R are now waiting for commands. 

\includegraphics[width=6.5in]{dbs4.png}

We can now run our {\bf parallel} code normally.
So, at the manager window, we type

\begin{verbatim}
> clusterEvalQ(cls,f(8))
\end{verbatim}

And {\it voila!}, we are in debug mode in the two cluster node windows,
with the familiar

\begin{verbatim}
Browse[]
\end{verbatim}

prompt:

\includegraphics[width=6.5in]{dbs5.png}

Again, we can then single-step etc. normally, say in the first cluster
node window:

\includegraphics[width=6.5in]{dbs6.png}

Last, when we are done with this debugging sequence, we can kill the
{\bf screen} invocations and so on, by calling {\bf killdebug()}:

\includegraphics[width=6.5in]{dbs7.png}

\end{document}


